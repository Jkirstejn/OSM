\documentclass[11pt,a4paper]{article}

\usepackage[utf8]{inputenc}
\usepackage[english]{babel}
\usepackage{listings}
\usepackage{color}
\usepackage{graphicx}
\usepackage{float}
\usepackage{amsmath}
\usepackage{qtree}
\usepackage{verbatim}
\usepackage{array}

\title{Operativsystemer og multiprogrammering \\ Handin 3}
\author{Malte Stær Nissen and \\ 
        Jacob Daniel Kirstejn Hansen}

\begin{document}
\maketitle

\tableofcontents
\newpage

\section{Overall assumptions}
We've made the following assumptions, on which our implementation depends on: 

\begin{itemize}
\item The programs are small enough to be run on the system without memory problems
\end{itemize}

\section{Design decisions}
<<<<<<< HEAD

In order to maintain atomic access to our locks, we chose to implement them by disabling interrupts and using spinlocks. This means that every time we acess the state variable of a given lock we first acquire the spinlock for the lock we wish to access. We then access the state of the lock and perform the desired changes followed by a release of the spinlock at the end. Furthermore we encapsulate the entire access to the spinlocks and the lock state with disabling of interrupts. We simply follow the guidelines given in the Buenos Roadmap, page 29.

We similarily disable interrupts in the implementation of the conditions. We don't use spinlocks in the implementation of the conditions though since it doesn't make any sense to do so.
=======
When we're implimenting the locks, we both disable interrupts in "lock\_reset"
to begin with and enable them again before we return. In "lock\_acquire" we
only disable interrupts when we're adding a process to the sleep que. It is the
responsibility of our "lock\_release" function to enable the interrupts again.
We've extended these decisions to the design of the conditional locks as well.
>>>>>>> 355567e35b6f0b1d48fa0f1093f215e5aeecd44d

We've chosen to give the syscalls for lock functions the hex values 0x301,
0x302 and 0x303. This means that the only syscalls starting with 0x3 are the
lock functions. To follow this decision we chose to give the conditional locks
hex values starting with 0x4, hence the values are 0x401, 0x402, 0x403 and
0x404.

\section{Important observations}
When we test out the conditions and locks using the test program given in
tests/threads\_ring.c, the program starts failing if we set it to run for too
many rounds. If we eg. set the number of rounds to 10, the program stops at the
7th round and stalls. We assume that this is because of a lack of memory and
hence our assumption about programs being small enough to run on the system
isn\'t met.

\section{List of changes}

\begin{itemize}
\item kernel/lock\_cond.h, kernel/lock\_cond.c: Locks and conditions
implemented.
\item kernel/module.mk: lock\_cond.c added to the list of files to be compiled.
\item proc/process.c, proc/process.h: Fork function implemented.
\item proc/syscall.c, proc/syscall.h: System calls for locks, conditions and
fork implemented. Unique hex values for systemcalls defined (in the .h file).
\item tests/lib.h, test/lib.c: System calls implemented here as well.
\item tests/threads.c, tests/threads\_locks.c, tests/threads\_ring.c: Tests for
fork, locks and conditions added.
\item compile: Bash script for cleaning everything up, compiling, deleting hdd,
creating new hdd and adding all tests programs.
\end{itemize}

\end{document}

