\documentclass[11pt,a4paper]{article}

\usepackage[utf8]{inputenc}
\usepackage[english]{babel}
\usepackage{listings}
\usepackage{color}
\usepackage{graphicx}
\usepackage{float}
\usepackage{amsmath}
\usepackage{qtree}
\usepackage{verbatim}
\usepackage{array}

\title{Operativsystemer og multiprogrammering \\ Handin 2}
\author{Malte Stær Nissen and \\ 
        Jacob Daniel Kirstejn Hansen}

\begin{document}
\maketitle

\tableofcontents
\newpage

\section{Overall assumptions}
We've made the following assumptions that our implementation of the different
functions depend on:

\begin{itemize}
    \item We've chosen to make a static sized bound of 32, on the number of
    processes we can have running at the same time.
    \item We've agreed to run a 1-to-1 strategy between user-level treads and
    kernel-level threads.
    \item We're representing alive and dead processes as enums instead of
    integers, so we didn\'t have too many integers in play to confuse.
	Free process slots are marked as free with an enum as well
	\item Every time we change something in the process table, we have made
	sure to disable interrupts and acquired the needed spinlock for the process table.
\end{itemize}

We choose to create a static array to represent our process table. The reason
is that the BUENOS kernel doesn't have a heap implemented, therefore we can't
dynamically allocate more memory for our array. Furthermore it's simpler to
work on a static array, because we want to save it across system calls and
processes.

By using a 1-to-1 strategy, we just create a thread aswell whenever we wish to initialize a new process, and associate it with the process.
This however isn't the case when we start up the first process.

\section{List of changes}

\begin{itemize}
    \item init\/main.c: Initialization of process table added to init\(\)
    function and function to start and link the process to the first thread
    implemented.
    \item kernel\/thread.h \& kernel\/thread.c: Function for getting a thread entry added
	\item proc\/process.c \/ proc\/process.h: Processes implemented
	\item proc\/syscall.c: System calls implemented using process functions
	\item kernel\/config.h: Boundary on max number of processes defined
\end{itemize}


\section{Syncronization}
We need to make sure, that our process table doesn't get corrupted when
performing changes on it. This is made sure by use of syncronization
principles. Whenever we need to perform changes on the process table we start
out by disabling interrupts. This is followed by an acquisition of a spinlock,
which is shared by all the functions, that performs changes on the process
table. After having initiated these precautions we can alternate the process
table according to our wishes without having to fear that some other process or
threads perform operations on the process table before we're done. When we are
done performing the desired changes to the process table, we start out by
releasing the spinlock and finish off by setting back the interrupt mask to
whatever it was before we disabled it.

\end{document}

