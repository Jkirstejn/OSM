\documentclass[11pt,a4paper]{article}

\usepackage[utf8]{inputenc}
\usepackage[english]{babel}
\usepackage{listings}
\usepackage{color}
\usepackage{graphicx}
\usepackage{float}
\usepackage{amsmath}
\usepackage{qtree}
\usepackage{verbatim}
\usepackage{array}

\title{Operativsystemer og multiprogrammering \\ Handin 5}
\author{Malte Stær Nissen and \\
        Jacob Daniel Kirstejn Hansen}

\begin{document}
\maketitle

\tableofcontents
\newpage

\section{Design decisions}
We're using \_tlb\_write\_random in the tlb\_lookup\_pagetable function,
instead of actually implementing a strategy to decide which entry should be
overwritten. To improve the performance of the operating system, one would
among other things implement a this strategy.

\section{Assumptions}
We assume that whenever we make a call to malloc, the size of the memory we
wish to allocate is never greater than the size of a page (4096 bytes). By
making this a assumption we simply need to free the entire page of an entry.
We're not handling fragmentation, but allocate new memory regardless of already
freed pages.

\section{How we would have implemented it}
We would've implemented the malloc and free functions by using a simple heap
along with a linked list containing information about the size of the different
entries. These sizes would've been entire pages so every time you wanted to
allocate new memory, at least one new page of 4k would've been allocated at the
end of the heap. Likewise we would've freed at least one page every time the
free function was called. This would've given quite a fragmented memory, but
would be a bit easier to implement than the way it should have been
implemented.

\section{How it should have been implemented}
Naturally it would make more sense to check the virtual memory for fragments of
free memory before blindly allocating atleast 1 more page each time we call
\texttt{malloc}. As a consequence of this, \texttt{free} shouldn't free an
entire page but only the entry block starting at the argument address.

\section{List of changes}
\begin{itemize}
	\item proc/exception.c, kernel/exception.c: Handling of exceptions EXCEPTION_TLBM, EXCEPTION_TLBL & EXCEPTION_TLBS forwarded to tlb_modified_exception(), tlb_load_exception() & tlb_store_exception().
	\iem vm/tlb.c: tlb_modified_exception(), tlb_load_exception() & tlb_store_exception() implemented along with the additional function tlb_lookup_pagetable to perform lookup in a pagetable. 
	\item thread.h, thread.c: Functions to handle allocation and freeing of memory pages implemented. Begun to implement malloc and free but didn't succeed. Constants defined in thread.h and dummy alignment for padding a thread entry to 64 bytes corrected.
	\item vm/vm.c: vm_unmap() implemented.
\end{itemize]

\end{document}

