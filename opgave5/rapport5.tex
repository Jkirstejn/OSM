\documentclass[11pt,a4paper]{article}

\usepackage[utf8]{inputenc}
\usepackage[english]{babel}
\usepackage{listings}
\usepackage{color}
\usepackage{graphicx}
\usepackage{float}
\usepackage{amsmath}
\usepackage{qtree}
\usepackage{verbatim}
\usepackage{array}

\title{Operativsystemer og multiprogrammering \\ Handin 5}
\author{Malte Stær Nissen and \\
        Jacob Daniel Kirstejn Hansen}

\begin{document}
\maketitle

\tableofcontents
\newpage

\section{Design decisions}
We're using \_tlb\_write\_random in the tlb\_lookup\_pagetable function,
instead of actually implementing a strategy to decide which entry should be
overwritten. To improve the performance of the operating system, one would
among other things implement a this strategy.

\section{Assumptions}
We assume that whenever we make a call to malloc, the size of the memory we
wish to allocate is never greater than the size of a page (4096 bytes). By
making this a assumption we simply need to free the entire page of an entry.
We're not handling fragmentation, but allocate new memory regardless of already
freed pages.

\section{How we would have implemented it}



\section{How it should have been implemented}
Naturally it would make more sense to check the virtual memory for fragments of
free memory before blindly allocating atleast 1 more page each time we call
\texttt{malloc}. As a consequence of this, \texttt{free} shouldn't free an
entire page but only the entry block starting at the argument address.

\end{document}

